% !TEX root = research_report.tex
%上の記述は消さないこと.章(分割ファイル)を増やしたときも忘れずに記述する.
%\chapter*{参考文献}

\begin{thebibliography}{99}
 \bibitem{asimo}``ASIMO SPECIAL SITE'': http://www.honda.co.jp/ASIMO/
 \bibitem{h6h7} 大学院情報理工学系研究科 知能機械情報学専攻 大学院情報学
	 環・学際情報学府 工学部 機械情報工学科 情報システム工学研究室
	 (JSK)``Perception-Action Integrated Humanoid Robot : H6 \&
	 H7'': http://www.jsk.t.u-tokyo.ac.jp/research/h6/H6\_H7.html
 \bibitem{hrp}独立行政法人産業技術総合研究所 ``Humanoid Robotics Project
	 - 人間協調・共存型ロボットシステム'':
	 http://www.mstc.or.jp/hrp/main.html
 \bibitem{qrio}``SONY Dream Robot QRIO'':
	 http://www.sony.co.jp/SonyInfo/QRIO/index.html
 \bibitem{kawato}``川人学習動態脳プロジェクト'':
	 http://www.kawato.jst.go.jp/DB/home-j.html
 \bibitem{Erbatur} 
	K. Erbatur, A. Okazaki, K. Obiya, T. Takahashi,
	A.Kawamura: ``A Study on the Zero Moment Point Measurement for Biped
	Walking Robots'', Proc. Int. Workshop on Advansed Motion
	 Control, 2002
 \bibitem{hashimoto} 
	橋本 浩一: ``ビジュアルサーボにおける予測と感度'', 計測と制御,
	 Vol. 40, No. 9, pp. 630-635, 2001
 \bibitem{hashimoto2} 
	K. Hasimoto, T. Kimoto, T. Ebine and
	H .Kimura: ``Manipulator Control with Image-Based Visual Servo'',
	IEEE ICRA, pp. 2267--2272, 1991

\end{thebibliography}

書き方は上記のものをまねすること.課題研究では1,2本,卒業研究では10本,専攻科特別研究では20本程
度の論文を読んでおくべきであることから,その本数文の参考文献を挙げる.なるべく,学会誌論文,講演
会論文などを参考文献とするほうがよい.大学内での卒業論文などは入手が困難である確率が高いため載せ
ない方がよい.

本文中で参照した順番に並べること.本文中で参照していないものは最後の方に書く.
